\chapter{Legal, Social, Ethical and Professional Issues}

The usage of smart contracts for managing employment relations marks a significant technological leap. However, this approach poses certain professional concerns that must be properly addressed. 

\section{Relevant Professional Matters}

The relevant professional matters could be categorised into integrity, competence, confidentiality, and regulations compliance.

\subsection{Integrity and Honesty}

The fine-tuning of the custom AI model involves no real personal data. In order to not compromise any sensitive information, training datasets were created using fuzzing techniques and creative input. The capabilities and limitations of the platform are represented truthfully, avoiding overstatements about its efficacy or security. As stated in the evaluation chapter, the software is not yet ready for widespread adoption and requires further extensions. The inherent security problems associated with the use of current technology, such as smart contracts, are recognised. As cryptography specialist and lecturer at King's College London, Luca Vigano points out, "A computer is only truly safe when it is turned off and disconnected from all networks, hence any software poses a risk". Therefore although smart contract technology is improving, it is not immune to flaws. Despite being safer than many technologies established in the recent decades, smart contracts, like any other system based on blockchain technology, face potential vulnerabilities such as the 51\% attack. This type of attack enables entities with sufficient computational capacity to bypass network protection and completely rewrite the data saved in a distributed ledger. As with well-known and trusted systems such as \textit{OpenSSH}, new vulnerabilities are identified on a regular basis, necessitating constant vigilance. In the context of this project, although the technology itself offers greater security, it is still in its nascent stage and not yet suitable for production without further improvements in security measures. Nevertheless, the smart contracts have been developed to manage financial transactions and employment relations with the highest degree of transparency and security possible at this stage for a bachelor's thesis.

\subsection{Competence}

A high degree of competency must be maintained throughout software development processes. This necessitates ongoing learning and adaptation to new security mechanisms, as well as a grasp of the changing legal consequences of smart contract adoption. As previously indicated, this document serves as a bachelor's thesis and does not claim to offer the best solutions; the identified limitations are clearly outlined. Additionally, it has been noted that the author is a student, not an industry professional. Consistent with the evaluations detailed in the earlier chapters, a commitment to continuous learning is adhered to.

\subsection{Confidentiality}

Strict adherence to confidentiality agreements is demanded when handling sensitive employment data. To avoid unauthorised access, it must be encrypted and stored securely. Despite not deploying smart contracts within the scope of the application, the software is designed to prepare them for future deployment, which could expose private data. Addressing this will be a priority for future software extensions.

\subsection{Compliance with Regulatory Laws}

Blockchain's decentralised nature provides protection against forgeries, but it also serves as a double-edged sword; since data is immutable, any breach has long-term ramifications for privacy. The developed software partially adheres to international data protection standards, such as the \textit{GDPR} in the EU, which govern how personal data is acquired, stored, and shared. Specifically, no personal data is exposed to unauthorised individuals within the scope of the backend software. Additionally, appropriate permissions for data storage on the server are obtained from users. However, as soon as users deploy the generated smart contracts, their data becomes exposed to any node within the deployed blockchain. 
    
In terms of employment Laws, the developed smart contracts were designed to comply with international ones, but there are inherent restrictions to their area of coverage, particularly in terms of minimum wage and overtime work. Limitations arise primarily because the AI model only serves to translate legal employment contracts into code. This, in turn, implies no control over the original data input, which may include salaries below the minimum wage requirement. Furthermore, the absence of a specific job performance metrics evaluation system causes concerns, as the software does not address overtime workload issues, potentially leading to non-compliance with employment standards.
    
Regarding intellectual property rights, throughout the training of the AI model, those of \textit{OpenAI} and others are respected so that the use external technology does not infringe upon the copyrights of others. Additionally, the legitimate rights of other third-party tools used during development are properly credited and utilised in accordance with their licensing agreements.

\section{Environmental Impact}

\paragraph{Negative Impacts}

Traditional blockchain systems, especially those that rely on proof-of-work mechanisms, are renowned for their high energy consumption. The mining process requires substantial computational power, which implies the usage of significant amounts of electricity. This substantial energy consumption contributes to increased carbon emissions, which is concerning in light of global climate change. Furthermore, the hardware-intensive computations required to validate smart contract states in the blockchain necessitate servers that often have short lifespans due to continuous operation, contributing to electronic waste.

\paragraph{Indirect Positive Impacts}

This software, by digitising contract management and execution, can reduce the dependency on paper, minimising deforestation and waste associated with paper manufacture and disposal. Additionally, as it automates some business processes, this can potentially reduce the need for physical transport and office spaces, thus lowering overall energy consumption and gas emissions.

\section{Social and Ethical Implications}

Given the experimental nature of the software, the likelihood of market adoption is extremely low at this stage. However, ethical reasons justify and require a hypothetical study of its potential consequences.

The automation of contract management and performance evaluation may minimise human control in HR operations, posing a risk of job displacement. Consequently, this leads to justified concerns about the dehumanisation of workplace relationships. Although the software was designed to automate manual workloads and provide better social security, unintended negative outcomes may still occur. Ethical considerations must address how society will manage the transition for workers who may lose their jobs to automation.

\section{Public Well-Being Impact}

The project promotes equal access to IT benefits by making the software accessible to a wide demographic, regardless of belief, race, disability, gender, or any other protected characteristics. Actions are taken in the public interest, maintaining transparency about the experimental nature of smart contracts and their legal recognition. Smart contracts can make employment contracts more accessible to a wider range of people, especially those in distant or underserved places, by eliminating the need for middlemen and lowering entry barriers. With transparent and enforceable contracts, employees can experience greater job security and clarity regarding employment terms, contributing to social stability. Additionally, the technology can help protect workers from exploitative practices by strictly enforcing the terms of employment, provided those terms are fairly applied and once the software is complete.

\section{Sustainability, Economic, and Commercial Factors}

By automating contract enforcement and management, smart contracts can enhance the long-term viability of enterprises by reducing operational risks and costs. The automation of contract execution through these technologies significantly lowers expenses associated with legal services, administrative processing, and dispute resolution. Furthermore, smart contracts enable businesses to function more efficiently across borders, allowing for market expansion and access to global talent pools without incurring the significant costs that are traditionally associated with such growth.

The distinct characteristics of smart contract technology also attract venture capitalists interested in technology-driven company models, stimulating economic growth. Furthermore, the capacity to automate and protect employment contracts can lead to the development of new business models, such as decentralised autonomous organisations or blockchain-based gig economy platforms.

However, as previously mentioned, the technology challenges traditional employment structures, potentially displacing administrative jobs, necessitating careful assessment of the broader employment impact.
