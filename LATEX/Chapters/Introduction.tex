\chapter{Introduction}

\section{Purpose and Scope}

The digital revolution is upending the way most industries operate, and agreement practices are no exception. Within this scenario of change, this dissertation proposes a new method to transform the management of such processes. The work, in collaboration with IBM, proposes the integration of advanced generative Artificial Intelligence (AI), using the \texttt{GPT-3.5} model, with the high-tech smart contract technology developed in Solidity. This integration seeks to automate the creation and maintenance of legal employment contracts.

Astonishingly, despite the existence of the required framework, there is currently no adequate software solution for handling employment processes. Conventional approaches mostly depend on human resources departments manually managing these operations, which can require an extensive amount of time and be prone to human error. These manual practices often lead to unfair decisions influenced by personal prejudices about what might benefit the business, rather than providing equitable treatment of employees \cite{Wax1998, SusanEtAl1995}. Notably, while there have been isolated attempts to address these challenges technologically, such as the effort by Giorgia Lallai, Andrea Pinna, Michele Marches, and Roberto Tonelli from the University of Cagliari \cite{lallaiETAL2020software}, their implementation had several limitations and remains the sole attempt of its kind found in academic research. This demonstrates the critical need for a more comprehensive solution. The effort of this work is directed towards solving at least part of the natural complexities of the traditional employment market, which are particularly problematic in online scenarios. Here, the absence of structured contract management frequently leads to deception, with employees completing their job duties but not receiving payment \cite{Howson2023, Shevchuk2018, Aleksynska2018, McGauran2016}. The principal objective of this study is to promote a more efficient, compliant, and user-oriented approach for combating these issues. Another critical component of the project is the development of a web platform, through which different kinds of regular legal employment contracts will be transformed into smart contracts. The scope of the project extends beyond automation; it focuses on securing both parties in agreements and eliminating the need for third-party intervention in conflict resolution.

\section{Achievement Plan}

The most important implementation goals include:

\begin{enumerate}
    \item \textbf{AI Model Development:} Fine-tune a generative AI model based on \textit{OpenAI's GPT 3.5 Turbo} to convert traditional legal employment contracts into smart ones, designed in the Solidity language.
    \item \textbf{Backend Platform:} Develop a backend platform based on \textit{Django REST API} that integrates with the AI model for the purpose of contract conversions, alongside functionalities for managing and controlling them. Moreover, the platform will host forums for discussions on contract-related topics.
    \item \textbf{User-Friendly Frontend:} Create a user-friendly frontend interface based on \textit{React Native} that would act as the main medium for customer interaction.
\end{enumerate} 

Summarising, the intent of the work is to attempt to answer the question of whether it is possible to automate employment interactions using smart contract technology. Additionally, the aim is to create infrastructure for these means. 

\section{Economic, Commercial, Customer, and Research Context}

In addition to achieving technical objectives, another primary aim is to commercialise the end product. The software will hold significant market potential if it successfully achieves the intended goals. The main target audience for this product includes freelance platforms that currently employ numerous individuals to resolve disputes between customers and freelancers, arising from disagreements and misunderstandings. Additionally, if the software evolves, there is a chance of it being adopted by on-site employers, or at least this could be the starting point for further progress in the field.

Beyond these objectives, another important goal of this research thesis is to explore potential applications of this technological stack in the lucrative \textit{Web3} environments. Specifically, scenarios involving parties agreeing on terms, where one side may attempt to defraud the other. This technology could ensure trust and enforce agreements between the collaborating parties, thereby revolutionising \textit{Web3} interactions and opening new avenues for its deployment across diverse sectors.

\section{Summary of the Work}

The following chapter examines the evolution of smart contracts, from Nick Szabo's ideas to their implementation on Ethereum, covering their applications in various sectors and discussing their limitations, such as legal challenges and scalability. Chapter 3 outlines the technical framework for the platform, integrating AI, Django Rest API, and React Native to handle contract conversions and user interactions efficiently. Chapter 4 details the platform's implementation, including AI training and app development challenges. Chapter 5 evaluates the platform's functionality through unit testing, a CI/CD pipeline, and feedback from fellow students. It highlights unimplemented features, the benefits of iterative development, and compares the achievements to the initial goals. The findings indicate partial success in automating and securing employment management, but reveal critical limitations due to unimplemented job performance metrics evaluation system and the realisation that the AI model, initially thought crucial, was unnecessary, impacting the project's full automation and efficiency goals. Chapter 6 discusses the ethical, legal, and environmental issues associated with using smart contracts in employment, outlining the most critical concerns such as integrity and honesty in AI data handling, adherence to confidentiality and regulatory laws, competence in development, and the potential negative environmental impacts of blockchain technology. The conclusion, Chapter 7, reviews the project's initial success and future research directions, including its application in \textit{Web3} fields.
