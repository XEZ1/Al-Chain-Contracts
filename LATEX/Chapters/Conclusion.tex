\chapter{Conclusion}

\section{Achievements and Conclusions}

The project has succeeded in achieving the following accomplishments:

\begin{enumerate}
    \item An AI model was fine-tuned to translate legal employment contracts into smart ones for autonomous execution on the blockchain.
    \item A backend API was established to manage interactions with the AI model and support additional functionalities such as user management, forum communications, convenient editors and means to interact with the platform.
    \item A practical mobile application was developed with an intuitive UI/UX design.
    \item Together with a successful CI/CD pipeline integration, 100\% test coverage was attained for the backend, frontend, and AI component.
    \item The system with preliminary functionalities was deployed, being ready for the initial alpha/beta testing phases.
\end{enumerate}

Apart form the implementation achievements, this dissertation has attracted the interest of investors from \texttt{Block Dojo}, a global venture builder. Their investment has highlighted the project's potential, while also marking the start of the next round of software development based on the dissertation's findings. \textit{Block Dojo} has acquired a 10\% equity in the project, offering significant financial and strategic support for its growth. Furthermore, \texttt{IBM} has been the main stakeholder in this research. Their assistance has helped in assuring compliance with industry norms and expectations. Recognising the original quality of the research, an invitation was made to present it at the annual King's College London \texttt{Research Showcase Conference}. Being one of only two undergraduate students selected for this event, the findings and implications of the work were shared with a broader academic and professional audience.

One of the main goals of this thesis was to study the applications of artificial intelligence and smart contracts to understand their potential in solving the problem of mutual trust on the web. Through the development and implementation process, the study sought to advance personal knowledge of these technologies and identify potential markets that could significantly benefit from their application.

The study showed that it is difficult to fully automate employment interactions using smart contracts. While the technology itself shows great potential, real-world automation is challenging to achieve due to factors such as market preparedness, regulatory compliance, and technological constraints. Although the current implementation is not yet ready for widespread production use, it has laid a solid foundation for further development. Moreover, the underlying technologies themselves have demonstrated enormous promise in a variety of sectors. Subsequent development in industries such as real estate and marketing could yield significant economic and operational efficiencies. It will likely involve advancing the technology and enhancing its security protocols.

\section{Future Directions of the Research}

Upon conducting the research, one potential monetisable field was discovered. In the conventional process, small businesses engage with influencers to market their products through direct negotiations over the web, agreeing on terms and settling payments upfront without signing any legal documents to hold parties liable. This process is rife with scams, where either advertisers pay for services that they do not receive, or influencers go unpaid after completing promotional work. Therefore, moving forward to a Master's degree, there is an option to continue this research by applying the technology to the more feasible field discussed above. The proposed solution involves:

\begin{itemize}
    \item Creating a platform where users can directly track their marketing offers, monitor their fulfilment, and make new offers. This platform would integrate with social media APIs, providing an overview of influencers' account statistics, including their average views, likes, comments per post, user engagement, and all other relevant information.
    \item Advertisers deposit payment into a smart contract.
    \item \textbf{Three-Part Authentication for Payment Release:} The smart contract designed for these transactions would require three signatures to release funds:
    \begin{enumerate}
        \item From the brand ambassador, confirming agreement to the terms.
        \item From the advertiser, who tops up the smart contract.
        \item From the backend system, which validates the completion of the advertisement criteria as stipulated in the contract.
    \end{enumerate}
    \item \textbf{Automated Verification via AI:} Upon notification from the influencer that the promotional content has been posted, the platform's backend retrieves the data and passes it to a fine-tuned AI model that analyses the content to ensure it meets all agreed-upon promotional standards and criteria. 
    \item \textbf{Secure Funds Transfer:} Once the AI system confirms that all conditions have been met, the smart contract automatically processes the payment.
    \item \textbf{Why this Software is Realisable:} With the original topic of the dissertation, a significant challenge was the verification of dynamic data, which is complex to compute and not universally available from all companies. On the contrary, with this software, the data can be easily tracked since all social media (SM) content, whether posts, stories, or videos, can be pulled from the APIs of these SM platforms and verified against the agreed conditions of the offer. For instance, it can easily be ensured that a post highlights the product positively and mentions its specific features through modern Large Language Models, which can efficiently perform this verification through their advanced contextual awareness.
\end{itemize}

In light of the feasibility and promising potential of this application, it has been decided in collaboration with \textit{Block Dojo} to proceed with the development of this platform for marketing as the primary focus for the next phase of research and development.

Alternatively, if progressing directly to a PhD, this research could be continued in its original topic, attempting to tackle the issue of automated job performance metrics tracking and evaluation.
